\documentclass{article}
\usepackage[UTF8,fontset=windowsnew,heading=true]{ctex} % 使用包的方式添加时推荐使用的参数
\usepackage{listings}
\usepackage{xcolor}
\definecolor{vgreen}{RGB}{104,180,104}
\definecolor{vblue}{RGB}{49,49,255}
\definecolor{vorange}{RGB}{255,143,102}

\lstdefinestyle{verilog-style}
{
    language=Verilog,
    breaklines=true,  %代码过长则换行
    % basicstyle=\small\ttfamily,
    % keywordstyle=\color{vblue},
    % identifierstyle=\color{black},
    % commentstyle=\color{vgreen},
    numbers=left,
    % numberstyle=\tiny\color{black},
    numberstyle= \small,%行号字体
    % numbersep=10pt,
    tabsize=4,
    moredelim=*[s][\colorIndex]{[}{]},
    literate=*{:}{:}1,
    rulesepcolor= \color{gray}, %代码块边框颜色
    frame=shadowbox%用方框框住代码块
}
    
\makeatletter
\newcommand*\@lbracket{[}
\newcommand*\@rbracket{]}
\newcommand*\@colon{:}
\newcommand*\colorIndex{%
    \edef\@temp{\the\lst@token}%
    \ifx\@temp\@lbracket \color{black}%
    \else\ifx\@temp\@rbracket \color{black}%
    \else\ifx\@temp\@colon \color{black}%
    \else \color{vorange}%
    \fi\fi\fi
}
\makeatother
    
\usepackage{trace}
\begin{document}
This is my first document.

Happy texing!
豫章故郡,洪都新府。星分翼轸,地接衡庐。襟三江而带五湖,控蛮荆而引瓯越。

\begin{lstlisting}[style={verilog-style}]
module Mixing {
    ///////// ADC /////////
    inout              ADC_CS_N,
    output             ADC_DIN,
    input              ADC_DOUT,
    output             ADC_SCLK,
    ///////// ADC /////////
    input              AUD_ADCDAT,
    inout              AUD_ADCLRCK,
    inout              AUD_BCLK,
    output             AUD_DACDAT,
    inout              AUD_DACLRCK,
    output             AUD_XCK,
    ///////// clocks /////////
    input              clock2_50,
    input              clock3_50,
    input              clock4_50,
    input              clock_50,
    ///////// HEX /////////
    output      [6:0]  HEX0,
    output      [6:0]  HEX1,
    output      [6:0]  HEX2,
    output      [6:0]  HEX3,
    output      [6:0]  HEX4,
    output      [6:0]  HEX5,
    ///////// FOO /////////
    output      [2]    FOO,
}
\end{lstlisting}

\end{document}